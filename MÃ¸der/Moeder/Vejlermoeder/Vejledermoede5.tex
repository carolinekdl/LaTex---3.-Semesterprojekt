\section{Vejledermøde 5}

\vspace{0.5 cm}
\textbf{Indkaldelse til vejledermøde \#5} \\

Dato: 12-10-2018 \\
Tid: 13:00 \\
Sted: 408E \\
Deltagere: () Til stede, (F) for sent, (A) meldt afbud, (U) Udeblev \\
Caroline(), Kajene(), Mathias(), Mikkel(), Nicolai(), Sarah(), Thea(F), Samuel() 

\vspace{0.1 cm}
\textbf{Dagsorden:}

\begin{enumerate}
	\item Valg af mødeleder
	\item Valg af referent
	\item Godkendelse af referat fra forrige møde
	\item Opfølgning af aktionspunkter fra forrige møde
	\item Spørgsmål til Samuel
	\item Gennemgang af tidsplan
	\item Nye aktionspunkter til næste møde
	\item Tidspunkt for næste møde
	\item Evt.
\end{enumerate}

\textbf{Referat:}

\begin{enumerate}
	\item Nicolai
	\item Thea
	\item Godkendt
	\item Aktionspunkter opfulgt
	\item \textbf{Leveringstid på print:} \\
	Det kommer an på hvor pæne de er. Den bedste måde at gøre det på er at bestille det inde på eurocircuits. 
	Leveringstid er ca. 2-3 uger (7 arbejdsdage plus levering).
	Hvis man er i tidsnød kan man få et “grimmere” print ved standard print - leveringstid 5 dage. \\
	\textbf{Domænemodel: } \\
	Er det et krav at lave en overordnet domænemodel, hvor hardware også er med? -
	Det er vores eget valg. Hvis vi kan se at det giver mening at have med, så tag det med. \\
	\textbf{Aktionspunkter: } \\
	Både og, blev brugt på 2. semester som forberedelse på scrum! På dette semester bruger vi pivotaltracker i stedet for. Det er ens eget ansvar at sørge for at man har opgaver fra uge til uge. 
	Det er fint at det er grupper, og ikke bare hver person, der har opgaverne inde på pivotaltracker. Det er dog vigtigt at alle har noget. 
	\item Tidsplanen følges ikke helt \\
	\item - \\
	\item Næste møde: 8-11-2018 kl. 12:00 \\ 
	\item -
	
\end{enumerate}

\clearpage