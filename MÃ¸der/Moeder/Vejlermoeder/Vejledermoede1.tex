\section{Vejledermøde 1}
\vspace{0.5 cm}
\textbf{Indkaldelse til vejledermøde \#1} \\

Dato: 10-09-2018 \\
Tid: 12:00 \\
Sted: 408E
Deltagere: () Til stede, (F) for sent, (A) meldt afbud, (U) Udeblev \\
Caroline(), Kajene(), Mathias(), Mikkel(), Nicolai(), Sarah(), Thea(), Samuel() 

\vspace{0.1 cm}
\textbf{Dagsorden:}

\begin{enumerate}
	\item Valg af mødeleder
	\item Valg af referent
	\item Godkendelse af referat fra forrige møde
	\item Opfølgning af aktionspunkter fra forrige møde
	\item Spørgsmål til Samuel
	\item Gennemgang af tidsplan
	\item Nye aktionspunkter til næste møde
	\item Tidspunkt for næste møde
	\item Evt.
\end{enumerate}

\textbf{Referat:}

\begin{enumerate}
	\item Sarah
	\item Caroline
	\item -
	\item -
	\item \textbf{Opdeling af opgaver:} \\ Okay at opdele projektet i to opdelinger - Hardware/software. Tidligere oplevet at hardware er færdig længe før software. Fint at opdele, men 4 på software - 3 på hardware.
	1 person - process, scrum mm. Må ikke udelukkende have disse opgaver! Vigtigt at alle har lavet noget på enten hardware eller software. \\
	
	\textbf{Scrum:} \\ Ikke nødvendigt med møde hver dag! Vigtigt at vurdere hvad der er nødvendigt for projektet så hvert møde, referat mm. har indhold og nytte for projektet. Scrum kan fylde rigtig meget og administrative opgaver kan tage tid. Vigtigt at det ikke “stjæler” nyttig tid fra projektet. Måske er møde 1 gang om ugen for sjældent.\\
	
	\textbf{Samarbejdskontrakt mm:} \\ Afbud til møde senest 3 timer før - Afbud/udeblivelse bør skrives ned i logbøgerne.
	Gentagne gange ved udeblivelse tages det op i gruppen i fællesskab.\\
	
	\textbf{Navngivning af dokumenter:} \\
	Vejledermøde \# \\
	Gruppemøde \# \\
	
	\textbf{Logbog:} \\ Vigtigt at nedskrive test detaljer - eks. hardware tests. Hvis ikke i logbog, så i testdokument. Logbog - ugentlig basis eller pr. gang? \\
	
	\textbf{Use cases:} \\ UC1: Kalibrering og nulpunktsjustering
	Kalibrering skal være en use case for sig. De gøres ikke samtidigt (kalibrering 1-2 gange årligt, nulpunktsjustering før hver måling)
	Nulpunktsjustering bør være første punkt på hovedscenarie i anden use case. \\

	UC2: Mål og vis blodtryk og puls
	Bør måske splittes op i flere. Hvis hovedscenarie giver anledning til for mange extensions er den for omfattende/rodet.\\
	
	UC3: Alarmér hvis blodtrykket overstiger grænseværdier
	Evt. ekstra use case: Justér grænseværdier
	Lyd og visuelt - vis alarm.\\
	
	UC4: Gem data
	Gem i fil
	Give besked hvis det ikke lykkedes at gemme.\\
	
	\textbf{Aktører:} \\ I stedet for virtuel patient: Blodtrykssensor \\
	
	\item Tidsplanen laves af Sarah snarest muligt \\
	\item Snakke om projektleder, scrummaster og ledere på sprint. \\
	Kravspecifikation og accepttest laves næsten færdig. Deadline 21. september, næste vejledermøde er 18. september. \\ Tilrette aktør/kontekst- og use case diagram. Use cases mm. med rette aktører.
	Navn på alle opgaver ved arbejde med kravspecifikation og accepttest. Sørg for, at alle har noget at arbejde med! \\
	
	\item Næste møde: Tirsdag d. 18-09-2018 kl. 12:00 \\ Næste møde igen: Tirsdag d. 25-09-2018 kl. 12:00 \\ Derefter fast møde hver torsdag kl. 12:00\\
	
	\item -
	
\end{enumerate}

\clearpage