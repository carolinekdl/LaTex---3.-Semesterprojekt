\section{Vejledermøde 8}

\vspace{0.5 cm}
\textbf{Indkaldelse til vejledermøde \#8} \\

Dato: 22-11-2018 \\
Tid: 12:00 \\
Sted: 408E \\
Deltagere: () Til stede, (F) for sent, (A) meldt afbud, (U) Udeblev \\
Caroline(), Kajene(A), Mathias(), Mikkel(), Nicolai(), Sarah(), Thea(), Samuel() 

\vspace{0.1 cm}
\textbf{Dagsorden:}

\begin{enumerate}
	\item Valg af mødeleder
	\item Valg af referent
	\item Godkendelse af referat fra forrige møde
	\item Opfølgning af aktionspunkter fra forrige møde
	\item Spørgsmål til Samuel
	\item Gennemgang af tidsplan
	\item Nye aktionspunkter til næste møde
	\item Tidspunkt for næste møde
	\item Evt.
\end{enumerate}

\textbf{Referat:}

\begin{enumerate}
	\item Sarah
	\item Caroline
	\item Godkendt
	\item - Pivotal Tracker
	\item \textbf{Kalibrering:} \\
	Spørg evt. Peter. 
	Kendt tryk, kendte millivolt. Manuel justering for at få et præcist, forventet resultat. Målepunkter på vandsøjlen. \\
	\textbf{Grænseværdi for blodtryk: } \\
	Ugyldig værdi: Negativ værdi, bogstaver - fysisk fejl, kan ikke lade sig gøre. Skal kun kunne taste positive numre ind. Værdier der er realistiske. Evt besked “er du sikker”? eller blot låse den til kun at tillade værdier inden for bestemte grænser. Evt. brug af sliders i steder - positive/negative ting ved begge. \\
	\textbf{Opstilling af bilag:} \\
	Hardware: Arkitektur: BDD/IBD og tabeller. Design: (evt. udregninger hér), implementering: Fysisk arbejde på fumlebræt og printplade \\
	Software: Arkitektur skal indeholde domænemodel og tomt klassediagram. \\
	\textbf{Digitalt filter:} \\
	Evt. vælge filtre fra DSB for at vise, at vi også kan lave andet end et midlingsfilter. \\
	\item Godkendt - Med ift. tidsplanen
	\item - 
	\item Næste møde: 29-11-2018 kl. 12:00 
	\item -
	
\end{enumerate}

\clearpage