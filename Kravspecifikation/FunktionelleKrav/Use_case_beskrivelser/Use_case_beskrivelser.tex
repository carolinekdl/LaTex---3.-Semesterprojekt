\section{Use Case beskrivelser}
Dette afsnit giver en kort beskrivelse af systemets Use Cases. Systemet har mange funktioner, herunder måling af puls og blodtryk samt vise det på brugergrænsefladen. Grafen for blodtryk vises kontinuert og værdi for systolisk-, diastolisk-, og median blodtryk samt puls vises i form af tal. Derudover kan systemet alarmere i henhold til punkt 2.2. Systemet kan gemme blodtryksværdierne, digital filter status, brugerens ID-nummer, patientens CPR-nummer, alarmer samt alarmbetingelser. Systemet har endvidere et digitalt filter, som kan slås til og fra på brugergrænsefladen. 
\subsection{Use Case 1: Mål og vis blodtryk og puls}
Formålet med denne use case er få en patients blodtryk visualiseret kontinuerligt samt vise værdierne for patientens systoliske-, diastoliske- og middelblodtryk samt puls. Før brugeren kan igangsætte måling af patientens blodtryk og puls skal der foretages en nulpunktsjustering. Dette gøres i forhold til det atmosfæriske tryk i rummet. Brugeren trykker på “Lav nulpunktsjustering”, hvor efter der laves en nulpunktsjustering. Derefter kan brugeren trykke på “start”, hvorefter målingen startes. Derefter kan brugeren slå det digitale filter til eller fra. Brugeren stopper målingen ved at trykke på “stop”.


\subsection{Use Case 2: Justér grænseværdier}
Formålet med denne use case er at justere grænseværdierne for puls samt systolisk- og diastolisk blodtryk. Der er pr. default valgt grænseværdier for blodtryk og puls som beskrevet i systembeskrivelsen. Hvis forholdene ændrer sig, så er det muligt for brugeren at justere i disse grænseværdier. Dette gøres ved at trykke på knappen “Indstil grænseværdi”, og derefter indtaste de ønskede værdier. 

\subsection{Use Case 3: Alarmering}
Formålet med denne use case er at der udløses en alarm i henhold til, det der er beskrevet i punkt 2.2. Brugeren kan vælge at gøre de auditive alarmer lydløse, eller stoppe alarmen helt. Dette gøres på brugergrænsefladen.
Dato og tid for alarmen, de tilknyttede alarm grænser, alarmbetingelsen, grænseværdierne som blev overskredet og alarmens prioritet logges i henhold til dokumentet “Standard 60601-1-8”.

\subsection{Use Case 4: Gem data}
Formålet med denne use case er at rådata, start- og slut tidspunkt, kalibreringsdato, kalibreringsværdi, nulpunktjusteringsværdi, patientens CPR-nummer og brugerens id gemmes i en fil. Det systoliske-, diastoliske-, middelblodtryk og puls gemmes ikke, da disse kan genskabes ud fra rådataen. Efter brugeren har foretaget en blodtryksmåling kan vedkommende vælge at gemme patientens data ved at trykke på knappen “gem”. Der åbnes et nyt vindue, hvor brugeren skal indtaste patientens CPR-nummer, personale ID og dato med tid. Derefter trykker brugeren på “gem” knappen på det nye vindue. Hvis CPR-nummeret ikke er korrekt, kommer der en besked om at brugeren skal tjekke patientens cpr-nummer. Hvis CPR-nummeret er korrekt trykkes der på “gem” knappen igen. Måligens data i form af systolisk- og diastolisk- blodtryk samt middelblodtrykket og puls gemmes i en fil. Kendes patientens CPR-nummer ikke under operation, evt. i forbindelse med en akut situation indtastes 000000-0000 i stedet for CPR-nummer, når der gemmes data.


\subsection{Use Case 5: Kalibrering}
Formålet med denne use case er at kalibre systemet, således et kendt tryk svarer til en kendt spænding. Kalibrering af systemet skal foretages ca. én gang årligt. En teknisk fagperson har ansvaret for kalibreringen af systemet og kan igangsætte kalibreringen ved at trykke på knappen kalibrering, der ses på brugergrænsefladen i punkt 3 “Skitse af brugergrænseflade”. Herefter foretages der 5 målinger med 5 forskellige trykværdier, som er henholdsvis 10 mmHg, 30 mmHg, 50 mmHg, 75 mmHg og 100 mmHg.
