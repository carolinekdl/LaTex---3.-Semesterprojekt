\chapter{Implementeringen}
Implementeringen
Hardware er implementeret således at den indeholder en PC, NI DAQ 6009( AD-converter), Sallen-Key Lavpas-filter, transducer og substraktor. På fumlebrættet er selve forstærkeren, filteret og substraktoren lavet. Disse tre dele er hver især blevet testet igennem, for at se om de gav de forventede værdier. Det hele blev sat sammen til et kredsløb og der blev testet for hele systemet. Dette blev gjort for at sikre, at det virkede som det skulle. Efter testningen af hele systemet, blev kredsløbet l lavet inde på multisim, for at se at det gav de forventede værdier endnu engang. Derefter blev multisimkredsløbet overført til ultiboard, hvor de tilhørende komponenter blev placeret og sat fast på printpladen.  Printet blev derefter sendt til afsted for at blive printet.
BILLEDE AF PRINTET OG ULTIBOARD

Systemet hænger sammen på den måde, at strømforsyningen gør strøm til substraktoren, transduceren,  og forstærkeren. Analog discovery forsyner spænding til forstærker, transducer og filter. Forstærkeren forstærker signalet fra transduceren, der sender signalet gennem filteret, som bliver filtreret. Signalet går igennem substraktoren, som neddæmper signalet til at gå fra 0-4 V til +- 2 V . Dette signal bliver sendt videre til DAQ’en, som omdannet det analoge signal til et digitalt signal, der sender signalet videre til PC’en. 

BILLEDE AF FUMLEBRÆT


