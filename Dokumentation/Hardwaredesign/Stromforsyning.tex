\section{Strømforsyning}
Til powersupply valgte vi i starten at det skulle være et USB, som hhv. giver 0 V og 5 V. Det valgte vi ikke at tage med da vi skal have at signalet skal trækkes ned til at gå fra -2,5 til 2,5 V. NI-DAQ 6009'en har en -+ indgang og et stel, og man kan sætte den til at kører i området +- 5 volt, og +- 2,5 volt. Hvis man kører signalet fra 0 til 5 volt og kører signalet ind på USB’et så vil man ikke få en god opløselighed. Man udnytter ikke dynamikken. Der har man 14 bit, og hvis man kun kører + på indgangen svarer det til at man kun kører på 13 bit. For at udnytte hele dynamikken i NI-DAQ 6009’en får man substraktoren ind, og der vil man få +- indgangen til at gå gå fra +- 2,5 V. Analog og substractor bruges derfor på signalet. Analogen bruges til at levere +- forsyning uden en inverter, der giver -5V, og substraktoren vil derved trække signalet ned, sådan at  man udnytter hele dynamikområdet på NI-DAQ 6009’en ved at den  går fra +- 2,5 V. Signalet skal gå fra +- for at udnytte dynamikken i NI-DAQ 6009’en. Den har en differential indgang, og den skal have en +- indgang. Derfor bruger vi analog for at lade den gå fra -+.
Strømforsyningen giver strøm til hhv. transduceren og forstærkeren med -+5 V.  Derudover giver strømforsyningen også stelforbindelse til hhv. transducer, forstærker, anti-aliaseringsfilteret og AD-converter. hej