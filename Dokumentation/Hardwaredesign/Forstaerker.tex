\section{Forstærker}
Beskrivelse af forstærker \\
Fra INA114 databladet ses en formel til udregning af komponentværdier for at skabe den ønskede gain. Da vi ønsker en gain på 640 udregnes komponentværdierne til følgende:

\subsection{Beregninger}
\vspace{0.5 cm}
\[ sensitivity = 5\cdot\frac{µV}{V\cdot mmHg} \]
\[ powersupply = 5V \]
\[ maximumPressure = 250 mmHg \]
\[ V_{out} =5\cdot\frac{µV}{V\cdot mmHg} \cdot 5V \cdot 250 mmHg = 6250 µV \]
\[ V_{out} =6,25 mV \]
\[ gain = \frac{4V}{V_{out}} = 640 \]
\[ G = 1+\frac{50k\Omega}{R_{G}} \]
\[ 640 = 1+\frac{50k\Omega}{R_{G}} \rightarrow R_{G}=\frac{5000}{639}\]
\[ R_{G} = 78,247 \Omega \]

Med ovenstående beregninger vurderes det, at vi skal bruge en modstand på ca. 78$\Omega$ for at forstærke signalet 640 gange.

\vspace{0.5 cm}

\subsection{Test af forstærker med spændingsdeler}
\vspace{0.2 cm}
For at teste forstærkerens virkning bruges en spændingsdeler til at gøre en spænding fra Analog Discovery mindre. Vi ønsker at teste, om forstærkeren forstærker signalet 640 gange. Komponenterne til spændingsdeleren udregnes til følgende:

Tilfældig modstand vælges
\[ R2=100 k\Omega \]

Tilfældig spænding vælges
\[ V1 = 4V \]

Ønsket nedforstærkning
\[ V2=0,00625V\]

R1 bestemmes:
\[ 0,00625V = 4V\cdot \frac{R1}{R1+R2}\rightarrow R1 = 156,4945 \Omega\]

\clearpage


\subsection{Test af forstærker med vandsøjle}

\clearpage

