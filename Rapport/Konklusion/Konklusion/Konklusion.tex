Vi kan konkludere, at det har været muligt at skabe en GUI, som viser de data, som knytter sig til en invasiv blodtryksmåling. Herunder en graf samt angivelse af puls, systolisk-, diastolisk- og middelblodtryk. GUI’en indeholder derudover de ønskede funktioner, der har været muligt at implementere på baggrund af vores prioriteringer, dvs. kravene ift. at kunne mute og unmute en alarm, foretage nulpunktsjustering og kalibrering.
Desværre lykkedes det os dog ikke at implementere en funktion, således at informationerne omkring blodtryksmålingen blev gemt i en fil. Der er derfor rig mulighed for videre arbejde ift. denne funktion. 

Det har ligeledes været muligt at skabe et elektronisk kredsløb, der kan forstærke signalet på de 12,5 mV fra tryktransduceren til 4 V. Derudover er det lykkedes at filtrer signalet vha. et aktivt lavpasfilter af typen Sallen Key, og dermed undgå aliasering af blodtrykssignalet. Ydermere er der implementeret en subtractor, således at vi kan udnytte det fulde af bits på vores AD-converten. 

Hovedparten af de valg, som vi har truffet ift. design og prioritering af funktioner, er sket på baggrund af den tænkte brugssituation. På en operationsstue, er det vigtigt, at blodtryksgrafen, pulsen, systolisk-, diastolisk- og middelblodtryk er tydelige. Dog er det ikke nødvendigt, at de kan ses på lang afstand. Dels fordi der er to skærme i systemet, og fordi den sundhedsfaglige person ikke vil være på lang afstand til skærmen. Derudover er der taget højde for, at den sundhedsfaglige ikke har øjnene på skærmene hele tiden, hvilket er årsagen til, at vi har valgt at give systemet alarmlyde i henhold til ISO-standarden 60601-1-8, når enten pulsen eller blodtrykket falder eller stiger i forhold til de indtastede grænseværdier. På denne måder kan det sundhedsfaglige personale gøres opmærksom på en muligvis farlig situation. 

Igennem processen, har vi som gruppe valgt at samarbejde om udarbejdelse af kravspecifikation og accepttest. Herved har hvert enkelt medlem hvert en del af de valg, som er blevet truffet og spillet en rolle ift. implementering og beskrivelse heraf. Til udarbejdelse af systemets hardware og software del, har gruppen delt sig i et software og hardware team for at skabe større effektivitet. For at de to teams har haft et billede af, hvad hinanden har været i gang med, har vi benyttet os af standup-møder to gange om ugen. Derudover har vi benyttet scrum til at få en struktur på de opgaver, der skulle løses, for at vi kunne udforme en prototype af vores system. Begge arbejdsformer samt udviklingsmodellen har skabt en fælles følelse af ejerskab over projektet og betydet, at alle i gruppen har haft mulighed for at udvikle sine kompetencer.
