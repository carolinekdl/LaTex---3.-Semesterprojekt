\section{Programmer}

Til projektet har vi brugt programmerne: 

\begin{itemize}
	\item LaTex
	\item Google Drive
	\item GitHub
	\item Visual Studio
	\item Multisim
	\item Ultiboard
	\item Waveforms
	\item Pivotal Tracker
	\item TeamGantt
	\item MatLab
	\item MathCad Prime 4.0
	\item EuroCircuits 
	\item Visio
\end{itemize}

Til hardware-delen har vi brugt MathCad Prime 4.0 til analyse-delen. I dette program har vi foretaget alle udregninger. Vi har brugt programmet Waveforms til realiseringen af vores byggede kredsløb - både på fumlebræt og på printpladen. 
Multisim og Ultiboard har vi brugt i forbindelse med vores design af vores printplade. Før printpladen blev sendt endeligt til print blev den testet gennem EuroCircuits, som også er leverandør af printpladen. 

Vores software er skrevet i Microsoft Visual Studio i sproget C\#. Det var et krav fra projektets vejledere at skrive koden i dette sprog. Vi har valgt at skrive koden til vores software i Microsoft Visual Studio da det er et program, vi har fået undervisning i på 1. og 2. semester. Som tilføjelelse til dette har vi valgt at implementere GitHub. Dette har muliggjort, at software gruppen har kunne arbejde på samme projekt på hver deres computer. Dette har optimeret software gruppens arbejde, så projektet var samlet fra start. Til software-delen har vi brugt MatLab til at lave alarmer.

Til at udforme diagrammerne indenfor arkitektur og design for hardware og software har vi brugt programmet Visio. 

Vi har også brugt GitHub i forbindelse med programmet LaTex. Dette har vi brugt fordi, at alle på den måde har kunnet skrive deres afsnit af rapporten og derefter synkronisere vha. GitHub, således at de resterende medlemmer i gruppen har alle afsnit.
Før vi valgte at kaste os ud i at lære at bruge LaTex brugte vi Google Docs under Google Drive til at skrive vores dokumenter. Her havde vi alle mulighed for at skrive, se og rette alle dokumenter og dette har fungeret godt som en start til projektet. På Google Drive har vi endvidere haft en samlet mappe til alle relevante dokumenter i forbindelse med projektet. 

Til selve styringen af projektet i forbindelde med scrum har vi brugt programmerne Pivotal Tracker og TeamGantt. Disse to programmer er beskrevet lidt nærmere i forrige afsnit \vref{sec:udvikling}. 
