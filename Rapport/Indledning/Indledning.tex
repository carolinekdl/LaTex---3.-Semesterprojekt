\chapter{Indledning}
Formålet med projektet er at udvikle et blodtryks målesystem, der kan angive blodtryk og puls. Når der måles blodtryk skelnes der mellem systolisk- og diastolisk blodtryk. Det systoliske blodtryk er det højeste af de to tryk, der måles. Dette tryk måles, når hjertet trækker sig sammen og arbejder hårdest. Det diastoliske blodtryk er det laveste tryk, der måles. Dette tryk måles, når hjertet slapper af og arbejder mindst. Det udviklede blodtryks system vil have justérbare grænseværdier, der gør, at værdier på for højt- eller for lavt blodtryk eller puls kan indstilles. Systemet vil alarmere auditivt og visuelt ved overskridelse af de satte grænseværdier.

!!!!!!!

Det udviklede blodtryks målesystem er udviklet med henblik på brug i operationsstuer på sygehuse. Systemet måler blodtrykket invasivt direkte i patientens arterier og egner sig specielt godt til større operationer med risiko for blødning eller traume. Den invasive måling af blodtrykket muliggør, at systemet kan vise blodtrykket kontinuert på en graf på systemets brugergrænseflade samt visualisere henholdsvis det systoliske-, diastoliske-, median blodtryk samt puls i form af tal. Man vil ved hjælp af den brugervenlige brugergrænseflade tydligt kunne se patientens blodtryk og puls og eventuelle fald- eller stigninger heraf.

Systemet fungerer således, at der før hver måling skal foretages en nulpunktsjustering for at sikre, at blodtrykket måles korrekt i forhold til det atmosfæriske tryk i den pågældende operationsstue. Brugeren/sundhedspersonalet foretaget nulpunktsjusteringen efter gældende vejledning. Én gang årligt vil det være nødvendigt, at en tekniker kalibrerer systemet for at sikre optimal virkning. Efter endt måling er det muligt at gemme opsamlet data under patientens CPR-nummer.

I forbindelse med projektet er der udviklet en prototype af en blodtryksmåler, der blandt andet indeholder en brugergrænseflade til at vise blodtryk og puls. Udover brugergrænsefladen vil systemet bestå af en række elementer, herunder: Væskefyldt kateter, tryktransducer, forstærker og antialiaseringsfilter samt AD-Converter. Systemets funktionalitet og opbygning er nærmere beskrevet i afsnit x ”systembeskrivelse”.  


