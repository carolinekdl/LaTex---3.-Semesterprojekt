\chapter{Indledning}
Vores fornemste opgave som sundhedsteknologer er at lave udstyr, som kan hjælpe fagpersoner
i sundhedssektoren. Især på sygehuse er der brug for mange sundhedsteknologier, som vi
skal være med til at udvikle og gøre bedre. I daglig klinisk praksis er der ofte behov for
kontinuert at monitorere patienters blodtryk invasivt, i særdeleshed på intensive afdelinger
samt operationsstuer, hvor blodtrykket er en vigtig parameter til monitorering af deres
helbredstilstand. Formålet med dette projekt har været at udvikle et system, der kan opfylde
netop dette behov. Vi har valgt at arbejde med projektet i forhold til den valgte brugssituation
på en operationsstue. Da systemet skal kunne bruges på operationsstuer, stiller dette selvfølgelig
mange krav, da det er livsnødvendigt, at systemet er funktionelt og intuitivt at bruge. Systemet
består både af hardware og software, som i sammenspil kan måle, processere og visualisere blodtryk
samt puls på en computerskærm. Monitorering af blodtrykket invasivt bruges på operationsstuer
og intensive afdelinger verden over. Derfor er standarden ISO 60601-1-8:2007 udarbejdet, der omhandler generelle krav, prøvninger og vejledninger for alarmsystemer i elektromedicinsk udstyr. 
Vi har valgt at implementere uddrag af standarden ISO 60601-1-8: 2007. De dele af standarden, vi har valgt at implementere kan læses i dokumentet alarmbeskrivelse. 

Projektet har givet os mulighed for at anvende vores viden fra fysiologi-kurser, elektronik-kurser,
signalbehandlingskursus og ikke mindst programmeringskurser. Teoretisk viden fra disse kurser
har dannet grundlag for, at vi har været i stand til at designe og udvikle systemet.

Når man skal måle blodtryk, skal man først og fremmest bruge en tryktransducer, som
måler signalet. Dette analoge signal skal behandles ved hjælp af hardware inden det bliver
digitaliseret. Når signalet er digitaliseret, kan det bruges i en software applikation, som sørger
for at lave beregninger på signalet, visualisere det, alarmere hvis der er pludselige ændringer og kontinuerligt give et overblik over patientens status. Dette leder frem til den konkrete
problemformulering, som vi har arbejdet ud fra.




