\section{Use Case beskrivelser}
Dette afsnit giver en kort beskrivelse af systemets Use Cases. For en yderligere og mere detaljeret beskrivelse af Use Cases henvises til dokumentet kravspecifikation[X]. I vores projekt har vi fået givet en række obligatoriske krav, som vores system skal kunne indeholde. Dette har mundet ud 5 use cases. Derudover har vi valgt at have 5 usecases, i stedet for ét samlet, grundet hvert medlem i software-gruppen skulle havde en del at arbejde med. Det har været mere overskueligt og enklere at dele use casene op på denne måde, da det gav et bedre overblik over systemet.
   
\subsection{Use Case 1: Mål og vis blodtryk og puls}
Før en måling kan igangsættes skal systemet nulpunktjusteres. Patientens målte blodtryk visualiseres kontinuert i form af en graf på brugergrænsefladen. Systolisk-, diastolisk-, median blodtryk og puls vises i form af tal ligeledes på brugergrænsefladen.
\subsection{Use Case 2: Justér grænseværdier}
Før og under en måling er det muligt at justere grænseværdier for blodtryk og puls. Default værdier for grænseværdierne findes i systembeskrivelsen i bilaget kravspecifikation afsnit xx.
\subsection{Use Case 3: Alarmering}
Før og under en måling skal systemet kunne alarmere ved eventuelle fejl eller ændringer. Systemet alarmerer i henhold til standarden og prioriteres efter \textit{high priority}, \textit{medium priority} eller \textit{low priority}.
\subsection{Use Case 4: Gem data}
Efter endt måling er det muligt at gemme patientents blodtryksmåling, digitalt filter status og alarmer i en fil. Personale ID, patientens CPR-nummer samt dato og tid indtastes. Kendes patientens CPR-nummer ikke indtastes "000000-0000".
\subsection{Use Case 5: Kalibrering}
Kalibrering af systemet skal foretages at teknisk personale. Kalibreringen foregår én gang årligt.
