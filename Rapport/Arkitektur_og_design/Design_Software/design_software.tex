\section{Softwaredesign}
Dette afsnit beskriver, hvordan softwarearkitekturen er realiseret, samt hvordan de forskellige lag og klasser kommunikerer med hinanden. Overordnet set er det tilstræbt at designe softwaren ud fra paradigmet om lav kobling. Dette er først og fremmest gjort ved, som beskrevet i arkitekturafsnittet, at dele de forskellige use cases op i egne klasser med interfaces imellem. Derudover er der brugt dependency injection, som ligeledes fremmer en lav kobling. Et knudepunkt i softwaren er klassen HovedmenuGUI, som har mange ansvar. Dette ansvar er forsøgt mindsket ved at lave en GuiFactory klasse, som sørger for at oprette instanser af de tre andre GUI'er. Dette sikrer også, at det ville være nemt at tilføje en anden GUI, hvis softwaren skal udvides med en ny funktionalitet. 

Kommunikationen mellem datalaget og logiklaget er designet ud fra producer-consumer princippet. Dette var en oplagt løsning, da vores måleenhed hele tiden producerer data, som skal bruges og behandles af controllerklasserne i logiklaget. En anden vigtig feature ved producer-consumer er, at en datakø er indbygget, som sørger for, at der ikke går måledata tabt mellem datalaget og logiklaget, som kører i hver sin tråd. Køen har vi realiseret ved at bruge AutoResetEvents, som forhindrer, at producertråden overskriver måledata, inden de er blevet consumed i logiklaget. På den måde undgås fejl i kommunikationen mellem datalaget og logiklaget. 

Mellem logiklaget og præsentationslaget var det i første omgang planen at bruge observer pattern. Det viste sig dog, at dette ikke var særlig hensigtsmæssigt, da der er forskel på, hvor tit blodtryksgrafen og beregningerne på hovedmenuen skal opdateres. Dette gjorde det vanskeligt at bruge observer standardmønstret. Der var behov for en mere generisk løsning. Kommunikationen mellem logiklaget og præsentationslaget blev i stedet løst ved at bruge events, som på mange måder minder om observer, men er en mere fleksibel metode, der er indbygget i .NET frameworket. Dette viste sig også at være en god løsning på kommunikation mellem alarmeringscontrolleren og hovedmenuen. 

