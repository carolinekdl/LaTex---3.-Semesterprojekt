\chapter*{Resumé}
%\addcontentsline{toc}{chapter}{Resumé}

Formålet med dette projekt er at designe og udvikle et invasivt blodtryksmålesystem, der består af en brugervenlig grænseflade. Projektet omfatter også arbejdet med at have et system, der skal kunne kalibrere, kontinuerligt vise blodtryk, forstyrret hjertefrekvens. Systemet skal kunne alarmere hvis blodtrykket stiger eller falder. Derudover består programmet af et digitalt filter til filtrering af blodtrykket.

Gruppen besluttede sig for at designe et system, der passer ind i en operationsstue. Denne brugssituation har haft stor indflydelse på gruppernes beslutningstagning om design og funktioner samt hvordan arbejdet er prioriteret. Brugeren af blodtrykssystemet, som er den sundhedsfaglig personale, kan også via grænsefladen gemme og hente data fra en privat database.
Arbejdet har været prioriteret baseret på MoSCoW-modellen [X]. Denne prototype opfylder derfor de foreskrevne must-have kriterier vedrørende: farve størrelse, størrelse af værdier og mm.


\clearpage