\section{Test af software}
Dette afsnit beskriver, hvordan softwaren er testet, samt hvordan dette er dokumenteret. Som udgangspunkt er ideen med softwaretest altid at lave unittest, integrationstest og til sidst systemtest. Som nævnt ganske kort i afsnittet software design, så viste det sig at være sværere at unitteste softwaren end forventet. Dette tyder på at selvom softwaren er designet ud fra ideen om lav kobling, så har dette ikke været opfyldt tilstrækkeligt. 
Følgende elementer er blevet unittestet: Pulsalgoritmen, blodtryksalgoritmen, det digitale filter og lineær regressionsberegning i forbindelse med kalibrering. Se dokumentet Softwaretest (eller unittest afhængig af hvad dokumentet hedder???) for dokumentation af disse tests. 
En af de udførte unittests er som nævnt test af det digitale filter. Dette blev gjort ved at lave et sinussignal med hvid støj, som blev sendt igennem filteret. Nedenstående figur viser sammenligning mellem det filtrede og ufiltrerede signal. 

\vspace{0.5 cm}
\begin{figure}[h!]
	\centering
	\includegraphics[width=0.55\linewidth]{Implementering_og_test/Software/ufiltreret}
	\label{fig:ufiltreret}
	\caption{Test af digitalt filter: Ufiltreret}
\end{figure}

\begin{figure}[h!]
	\centering
	\includegraphics[width=0.54\linewidth]{Implementering_og_test/Software/filtreret}
	\label{fig:filtreret}
	\caption{Test af digitalt filter: Filtreret}
\end{figure}

Som nævnt i indledningen var det meget svært at lave unittest på vores use cases. Det endte derfor med at langt det meste softwaretest har været integrationstest, hvor hele use casen er testet på én gang. Dette blev gjort ved hjælp af debuggeren i Visual Studio. Ved at sætte breakpoints i koden og følge de forskellige kald ned gennem koden kunne fejl lokaliseres på denne måde. Problemet ved at bruge debuggeren og ikke skrive en selvstændig testsuite er en mangelfuld dokumentation af disse tests. Dette giver sig til udtryk i softwaretest (eller hvad det hedder) dokumentet, som kun indeholder de beskrevne unittests og ikke integrationstest. Dette kom også til udtryk under udførelsen af accepttest, hvor flere tests ikke opførte sig som forventet, hvilket igen indikerer en utilstrækkelig systematisk tilgang til test. 
