\section{Softwareimplementering}
I implementeringsfasen startede gruppen ud med at lave et ”skelet” for programmeringen ud fra vores klassediagram, så det var opdelt og klar til at blive kodet efter 3 lags-modellen. Da vi blev klar til at gå i gang med at kode begav vi os til at uddelegere forskellige programmeringsopgaver. Hvert medlem startede med at påtage sig nogle små opgaver, som var nogle basale ting. Efter udførslen af små-opgaverne blev der uddelegeret use cases, som man fik ansvaret for, at var klar og funktionel til deadline. Herefter blev de forskellige use cases sat sammen, og testet på tværs af lagene ud fra programmets funktionalitet, der er defineret i kravspecifikationen. 

Hele vores system er skrev i programmeringssproget er C# og er skrevet vha. programmerings programmet Visual Studios 2017.  Vi har gennem 1., 2. og 3. semester skrevet i sproget C# i undervisningen, og derfor har det været oplagt at skrive i dette sprog. Alle i gruppen har downloaded en udvidelse til Visual Studios kaldet ”GitHub”, hvilket har gjort det muligt at alle gruppens medlemmer har kunne sidde og arbejde i det samme dokument. Dette har gjort programmeringsprocessen en hel del mere overskuelig idet vi får alle dele til at spille sammen med det samme, og ikke sætter det tilsammen til sidst, hvro der kan forkomme et stort kaos. 
\section{Test af software}
\subsection{Modultest}
\subsection{Integrationstest}
