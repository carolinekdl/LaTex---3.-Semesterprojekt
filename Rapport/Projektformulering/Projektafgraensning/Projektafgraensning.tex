\section{Afgrænsning}
I kravspecifikationen er der beskrevet de ikke-funktionelle krav vha. MoSCoW- modellen. 

INDSÆT BILLEDE HER

I ovenstående tabel ses vores de MoSCoW krav, som er implementeret på vores system.

Vores fokus har været invasiv blodtryksmålingen, og derfor har vi valgt ikke at prioritere at
implementere fx EKG-måling. Vores won't have krav lyder derfor således, ”Systemet kan ikke
vise andet end blodtryksmålingen samt angive puls, diastolisk/systolisk blodtryk (F)”.

De softwaremæssige afgrænsninger i systemet er blandt andet, at der ikke kan hentes, samt åbnes
en tidligere måling fra en fil. Herudover har vi valgt at gemme målingens værdier i en fil, frem
for i en database. Ift. brugssituationen, ville det give god mening at benytte både en fil og en
database, da data kan tilgås og ses fra flere enheder. Eftersom det at gemme i en fil er et must
have krav, er det blevet prioriteret højere at opfylde dette end could have kravet om at gemme i
en database. Yderligere er et could have krav, at systemet kunne have et lydsignal ved afsluttet
blodtryksmåling, dette blev dog ikke prioriteret, da det kun var et could have krav. Derimod er
det prioriteret, at systemet skal kunne beregne og vise middelblodtrykket på GUI'en. Dette er et
must have krav, da man gennem en operation skal have en indikation på, hvordan blodtrykket
for patienten ligger. Systemet er en prototype, hvor softwaren kræver en PC som kan køre Visual Studio, hvor det til fremtidigt arbejde ville være oplagt at lægge softwaren på en mikroprocessor.

Det er nødvendigt at man manuelt skal foretage en nulpunktsjustering inden hver måling
igangsættes, for at sikre sig at man ikke måler det atmosfæriske tryk med i beregningerne. Sker dette ville vores system ikke kunne registrere værdierne, da vi har bygget systemet efter at kunne håndtere et max tryk på 250 mmHg. Derfor er dette et must krav. Endnu et must krav er, at der årligt skal laves en kalibrering af systemet, for at sikre en vedvarende præcision af systemet. Dette
foretages af en teknisk person, som sørger for at et kendt tryk svarer til en kendt spænding,
således systemet måler korrekt. Hardwaren er følsom overfor vand og støv, da der ikke er prioriteret at bygge en boks til at skjule printpladen. Derudover kræver hardwaren at Analog Discovery bruges som power supply.

Der er valgt en række værdier ifm. udviklingen af hardwaren, hvilket gør at den er begrænset
i forhold til hvilke størrelser af værdier den kan håndtere. Forstærkeren kræver specifikke
modstande, for at operationsforstærkeren skal forstærke op til de 4 V, som vi har valgt vores
system skal kunne. Derudover skal systemet benytte en tryktransducer med en sensitivitet
på 5,0 μV/(V*mmHg), da forstærkningsfakotoren er beregnet ud fra denne, og dermed er
modstandsvalget afhængig af en tryktransducer med en sensitivitet på 5,0 μV/(V*mmHg).

\clearpage