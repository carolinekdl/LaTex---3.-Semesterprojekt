\section{Afgrænsning}
I kravspecifikationen er der beskrevet de ikke-funktionelle krav vha. MoSCoW- modellen. 

INDSÆT BILLEDE HER

I ovenstående tabel ses nogle af vores MoSCoW krav, som er implementeret i vores system. For flere must have krav henvises der til kravsspecifikationen. Disse er valgt ikke at blive repræsenteret i tabellen, da disse testes gennem use casene.

Vores fokus har været invasiv blodtryksmåling, og derfor har vi valgt ikke at prioritere at
implementere fx EKG-måling. Vores won't have krav lyder derfor således, ”Systemet kan ikke
vise andet end blodtryksmålingen samt angive puls, diastolisk/systolisk blodtryk (F)”.


De softwaremæssige afgrænsninger i systemet er blandt andet, at der ikke kan hentes, samt åbnes
en tidligere måling fra en fil. Det vil i forhold til brugssituationen give god mening både at benytte en fil og en database til dette, da data kan tilgås og ses fra flere enheder. Dog eftersom det at gemme i en fil er et must have krav, er det blevet prioriteret højere at få denne funktion opfyldt end could have kravet om at gemme i en database. \newline
Yderligere er et could have krav, at systemet kunne have et lydsignal ved afsluttet blodtryksmåling, dette blev dog ikke prioriteret, da vi fandt andre alarmer mere relevante.
Systemet er en prototype, hvor softwaren kræver en PC som kan køre Visual Studio, og kan derfor ikke køres på andre enheder. \newline
Systemet kræver, at man manuelt foretager en nulpunktsjustering inden hver måling igangsættes, for at sikre sig at man ikke måler det atmosfæriske tryk med i beregningerne. Sker dette ville vores system ikke kunne beregne de korrekte værdier, hvilket gør dette krav til et must have. Endnu et must krav er, at der årligt skal laves en kalibrering af systemet, for at sikre en vedvarende præcision af systemet. Dette foretages af en teknisk person, som sørger for at et kendt tryk svarer til en kendt spænding, således systemet måler korrekt.


Hardwaren er følsom overfor vand og støv, da der ikke er prioriteret at bygge en boks til at skjule printpladen. Derudover kræver hardwaren at Analog Discovery bruges som power supply, da den leverer både -5 V, 5 V, 2 V samt stelforbindelse til printpladen. \newline
Der er valgt en række værdier ifm. udviklingen af hardwaren, hvilket gør at den er begrænset
i forhold til hvilke størrelser af spændinger den kan håndtere. Forstærkeren kræver specifikke
modstande, for at operationsforstærkeren skal forstærke op til de 4 V, som vi har valgt vores
system skal kunne, for at kunne udnytte alle 14 bits på AD-converteren. \newline
Derudover skal systemet benytte en tryktransducer med en sensitivitet på 5,0 μV/(V*mmHg), da forstærkningsfaktoren er beregnet ud fra denne, og dermed er modstandsvalget afhængig af en tryktransducer med en tilsvarende sensitivitet.

\clearpage