\section{Krav}
For ethvert system er der opstillet en række funktionelle og ikke-funktionelle krav for at sikre at kunden og udvikleren er enige om, hvilke funktionaliteter systemet skal have samt systemets kapacitet og betingelser.

For dette system er der tilknyttet fem aktører; to primære og tre sekundære. Den ene primære aktør er en bruger, som betjener blodtryksmåleren. Den anden primære aktør er det tekniske personale, som skal foretage kalibreringen af systemet. De sekundære aktører er blodtrykssensoren, som er et interface for patienten som er koblet til systemet, bluetooth højttaleren, der bruges til at afspille alarmlydene, samt storskærmen, der skal bruges til at vise brugergrænsefladen meget større, så den er let at se for alle på operationsstuen.

Aktørerne interagerer med systemet, hvilket opstiller 5 use cases (funktionelle krav). UC1 er hovedscenariet idet, der måles og vises blodtryk og puls, og angiver.  UC2 omhandler om at justér grænseværdier for puls samt systolisk- og diastolisk blodtryk. UC3 omhandler alarmeringen ift. ændringer i blodtryk samt puls eller ved fejl. UC4 omhandler funktionen at gemme patientens CPR, de opsamlede blodtryksdata samt puls gemmes i en privat database. UC5 omhandler kalibreringen af systemet. For yderligere detaljer se kravspecifikationen.

Når der opstilles en række use cases, opstilles der samtidig en række ikke-funktionelle krav, der skal overholdes for at use casene kan udføres optimalt. Den samlede oversigt over de ikke-funktionelle krav ses i kravspecifikationen. Her er der valgt at beskrive tre funktionelle krav der har fået prioriteringen must have ift. MoSCOW-modellen.

\begin{itemize}
\item Systemet skal have en brugergrænseflade \\
\item Systemets GUI skal alarmere i henhold til punkt 1.12.4, hvis pulsen
falder til under 60 slag i minuttet og overstiger 120 slag i minuttet  og at Systemet skal
alarmere i henhold til punkt 12.1.4, hvis blodtrykket falder til under 60
diastolisk eller stiger til over 160 diastolisk. \\
\item Diagrammet samt de viste værdier for puls skal kunne læses på op til 1,5 meters afstand af person uden synshandicap, eller som benytter korrekt korrigerende midler, og må ikke indeholde farver som påvirker folk med farveblindhed. 
\end{itemize}

Uden det ikke-funktionelle krav vedrørende brugergrænsefladen ville udviklerne kunne vise de opsamlede data som de ville, og ikke som kunden ønsker det. I dokumentet kravsspecifikation i bilaget ses flere ikke-funktionelle krav opstillet med fokus på brugergrænsefladen, så den er beskrevet så specifikt som muligt. \\


Det ikke-funktionelle krav, der henvender sig til alarmeringen er meget vigtig, da brugeren skal reagere hurtigt hvis patientens blodtryk eller puls falder eller stiger, da det kan være livstruende.

I henhold til brugssituationen i en operationsstue kan det ske, at brugeren måler patientens blodtryk og samtidig skal ordne noget ved underekstremiteterne, og dermed skal kunne læse værdien for puls og blodtryk på 1,5 meters afstand. Ligeledes skal der tages højde for at brugeren kan være farveblind, og dermed skal der benyttes andre farver end rød og grøn.\\