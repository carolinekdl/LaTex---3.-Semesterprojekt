\section{Krav}
Vi har for vores system, valgt at opstille en række funktionelle og ikke-funktionelle krav for at sikre at kunden og udvikleren er enige om, hvilke funktionaliteter systemet skal have samt systemets kapacitet og betingelser.

AKTØRKONTEKSTDIAGRAM SÆTTES IND HER


For dette system er der tilknyttet fem aktører; to primære og tre sekundære. Den ene primære aktør er det sundhedsfaglige personale, som betjener blodtryksmåleren. Den anden primære aktør er det tekniske personale, som skal foretage kalibreringen af systemet. De sekundære aktører er tryktransduceren, bluetooth højttaleren og skærmen. Tryktransduceren er et interface for patienten, som er koblet til systemet. Bluetooth højttaleren bruges til at afspille alarmlydene, og storskærmen skal bruges til at vise brugergrænsefladen meget større, således den er let at se for alle på operationsstuen.


Aktørerne interagerer med systemet, hvilket opstiller 5 use cases (funktionelle krav). UC1 er hovedscenariet idet, der måles og vises blodtryk og puls.  UC2 omhandler justeringen grænseværdier for puls samt systolisk- og diastolisk blodtryk. UC3 omhandler alarmeringen ift. ændringer i blodtryk samt puls. UC4 omhandler funktionen at gemme patientens CPR, de opsamlede blodtryksdata samt puls gemmes i en fil. UC5 omhandler kalibreringen af systemet. For yderligere detaljer se kravspecifikationen[X]. 

Når der opstilles en række use cases, opstilles der samtidig en række ikke-funktionelle krav, der skal overholdes for at use casene kan udføres optimalt. Den samlede oversigt over de ikke-funktionelle krav ses i kravspecifikationen[X]. Her er der valgt at beskrive fire funktionelle krav der har fået prioriteringen must have ift. MoSCOW-modellen (Must, Should, Could og Would).

ÆNDR TALLENE TIL KRAV

\begin{enumerate}
\item Den skærm, der skal benyttes af operatøren, er den skræm, hvor man interagere med systemet \\
\item Den skærm, der skal benyttes af observatøren, er den skærm, hvor man observere grafen samt værdierne for puls, systolisk-, diastolisk- og middelblodtryk
\item Systemet skal alarmere i henhold til punkt 12.1.4, hvis blodtrykket falder til under 60 diastolisk eller stiger til over 160 diastolisk. \\
\item Diagrammet samt de viste værdier for puls skal kunne læses på op til 0,5 meters afstand af person uden synshandicap, eller som benytter korrekt korrigerende midler, og farverne på grafen skal skelnes af folk med farveblindhed. 
\item Systemet skal kunne måle og vise et blodtryk på 250 mmHg efter nulpunktsjustering. 
\end{enumerate}

Uden det ikke-funktionelle krav vedrørende brugergrænsefladen ville udviklerne kunne vise de opsamlede data som de ville, og ikke som kunden ønsker det. I dokumentet kravsspecifikation i bilaget ses flere ikke-funktionelle krav opstillet med fokus på brugergrænsefladen, så den er beskrevet så specifikt som muligt. \\


De ikke-funktionelle krav, der henvender sig til alarmeringen er meget vigtig, da brugeren skal reagere hurtigt hvis patientens blodtryk eller puls falder eller stiger, da det kan være livstruende.

I henhold til brugssituationen i en operationsstue kan det ske, at brugeren måler patientens blodtryk og samtidig skal ordne noget ved underekstremiteterne, og dermed skal kunne læse værdien for puls samt blodtryk på 0,5 meters afstand. Ligeledes skal der tages højde for at brugeren kan være farveblind, og derfor skal der overvejes valget for farverne der bruges til graferne.

Grunden til, at blodtrykket er valgt til at være 250 mmHg ved atmosfærisk tryk, er valgt netop fordi, at vores (hardware, i form af forstærker), maksimalt kan måle op til denne værdi. Derudover er tallet, 250 mmHg, valgt med henblik på, at det ikke var realistisk at måle et blodtryk højere end dette.\\