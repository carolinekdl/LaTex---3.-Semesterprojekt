\section{Blodtryksmåling}
\vspace{0.1 cm}
Blodtryk kan måles invasivt eller non-invasivt. Blodtrykket måles enklest ved den non-invasive metode, hvor man eksempelvis kan bruge en oppustelig blodtryksmanchet tilsluttet et manometer. Proceduren er hurtig og smertefri.
Den invasive metode kræver indlæggelse af en kanyle i et af kroppens arterier. Metoden kan derfor være mere besværlig end ved den non-invasive blodtryksmåling. Den invasive metode kan med fordel bruges hos meget syge patienter eller på operationsstuer, hvor patienten enten bløder meget eller er i risiko for at komme til at bløde meget.  