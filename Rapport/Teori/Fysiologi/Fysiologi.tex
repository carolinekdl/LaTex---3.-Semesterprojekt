\section{Fysiologi}

Hjertet består af to halvdele og er adskilt af en kraftig skillevæg. Ud over de to hjertehalvdele består hjertet også af andre elementer, hér vigtigt at nævne hjerteklapperne, som sørger for, at blodet kun kan løbe én vej. Hjertet har fire hjerteklapper: To AV-klapper, aortaklappen og pulmonalklappen. Hjerteklappernes åbning og lukning bestemmes af trykket på henholdsvis den ene og den anden side. Højre side af hjertet pumper blod ud i lungekredsløbet, også kaldt det lille kredsløb, mens venstre side af hjertet pumper blod ud i legemskredsløbet, også kaldt det store kredsløb. Hjertets pumpefunktionen er en vigtig funktion, der har til hovedopgave at transportere blod rundt i kroppen.

\vspace{0.125 cm}
For at hjertet kan transportere blodet rundt til hele kroppen kræver det et tryk, der dannes ved, at hjertet trækker sig sammen med jævne mellemrum. Når hjertet slapper af er blodtrykket lavest - denne fase i hjertets cyklus kaldes diastolen, deraf kommer det diastoliske blodtryk. Når hjertet istedet kontraherer sig er blodtrykket højst - denne fase i hjertets cyklus kaldes systolen, deraf kommer det systoliske blodtryk.

\vspace{0.125 cm}
Diastolen starter, når trykket i ventriklerne er lavere end i arterierne. Dette medfører, at AV-klappen åbnes, så hjertet kan fyldes med blod. Under diastolen er aortaklappen lukket.

\vspace{0.125 cm}
Systolen starter, når trykket i ventriklerne igen overstiger trykket i arterierne. AV-klapperne lukkes for at forhindre tilbagestrømning af blodet, der derved vil løbe i den forkerte retning. Når trykket i venstre ventrikel overstiger trykket i aorta åbnes aortaklappen og blodet vil strømme ud. 

\vspace{0.1 cm}
Blodtryk angives normalvis i mmHg. Er blodtrykket 100 mmHg betyder det, at blodtrykket er 100 mmHg højere end det atmosfæriske tryk. Ved blodtryksmåling foretaget i hjertehøjde siges det, at et almindeligt blodtryk hos en ung voksen der slapper af ligger på ca. 120/70 mmHg. 120 angiver det systoliske blodtryk, mens 70 angiver det diastoliske blodtryk..