\section{Projektadministration}
Til dette projekt har vi valgt at starte ud med at benytte Google Docs som konfigurationsstyring. På Google Docs har vi haft forskellige mapper, hvori vi har lagt projektets dokumentation. Google Docs har den fordel, at man kan se, hvem der har oprettet de forskellige filer, hvem der sidst har rettet i dokumentet samt det, at man kan skrive flere i det samme dokument på én gang.

Gruppen benyttet sig af en fælles logbog, som der er blevet skrevet i hver uge. Her er der blevet beskrevet hvilke beslutninger vi har truffet og hvorfor, hvad de forskellige gruppemedlemmer har arbejdet med, om vi er stødt på problemer samt andet. På den måde har vi fået en opsamling hver uge på de forskellige opgaver og kunnet se, hvad de forskellige gruppemedlemmer har arbejdet på. Selve logbogen kan se i bilaget.

Til den interne kommunikation har vi benyttet os af vores fælles facebook gruppe, hvor vi har kommunikeret omkring mødetidspunkter, arbejdsopgaver og afbud til arbejdsgange og møder. Facebook har den funktion, at man på et opslag i gruppen kan se, hvem af gruppens medlemmer, der har set opslaget, og på den måde har man kunnet holde øje med, om alle har set opslaget og dermed har fået informationen. Dette har været med til, at hele gruppen har været inde i, hvor langt vi har været i processen.

Vi har i år valgt at bruge LaTex til at skrive rapport i, således at  opsætningen af vores rapport kommer til at se professionelt og flot ud. Da vi først har skulle sætte os ind i brugen af LaTex har vi skrevet alt brødteksten i Google Docs som beskrevet oven for og til sidst ført alle dokumenter ind i LaTex.

\vspace{0.3 cm}

