\section{Arbejdsfordeling}
Vi har i gruppen valgt at opdele projektet i to dele - Hardware og software. Der er en meget naturlig opdeling, da projektet er stort om omfattende og der derfor ikke vil være tid til, at gruppemedlemmerne kan sætte sig ind i det hele. 
Vi har opdelt grupperne med tre på hardwaredelen og 4 på softwaredelen alt efter arbejdsbyrden. 

Hardware-gruppen består af Caroline, Mikkel og Kajene.\\
Software-gruppen består af Thea, Mathias, Nicolai og Sarah.

I starten af projektet besluttede Sarah, at hun gerne ville arbejde som flyver i projektet og hjælpe til i den gruppe, der havde mest brug for det. I starten hjalp hun hardware-gruppen med at komme i gang, men er senere i projektet blevet en fast del af software-gruppen. Udover at have skiftet position i projektforløbet har Sarah også haft rollen som projektleder og Scrummaster og det har derfor automatisk heller ikke været muligt at følge med i begge dele af projektet på lige fod som i starten. Selve udviklingsforløbet har fyldt meget og software-gruppen har også haft nok at se til.

Opdelingen har fungeret godt fordi man har kunnet fordybe sig inden for et emne og dermed komme dybere ned i stoffet. På den anden side har det dog også til tider været frustrerende fordi det har været lidt svært at følge med i, hvad det andet team har arbejdet med. De ugentlige møder har dog været med til at vi har kunnet lave en opsamling på gruppen. 

Sarah har udover at være projektleder og scrummaster haft ansvaret for at udarbejde dagsorden til gruppemøder samt opdatere gruppens tidsplan på TeamGantt.

Caroline har haft ansvaret for at udarbejde dagsorden til vejledermøder

Thea har haft ansvaret for eventuelle konflikter

Grundet at hardware-gruppen har haft lidt overskydende tid til slut i projektet har Caroline og Kajene stået for en stor del af rapportskrivningen. Mikkel har hjulpet software.
 
\subsection{Skema over arbejdsfordeling}
På nedenstående skema ses arbejdsfordelingen med primære og sekundære roller til samtlige punkter i projektet.

