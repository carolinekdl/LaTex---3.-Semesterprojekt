\section{Møder}
\subsection{Gruppemøder}
I henhold til samarbejdskontrakten er det fra projektets start aftalt, at gruppen holder gruppemøde hver onsdag fra 8:15-9:50. Dette møde skal bruges til at følge op på forrige uges sprint, planlægge sprint for den kommende uge samt fordele arbejdsopgaver. Gruppemødet er også brugt til at snakke sammen på tværs af opdelingen mellem hardware og software, så begge dele af gruppen føler de har en forståelse for projektet som en helhed.

Med brugen af scrum som arbejdsmetode har vi valgt at lave to ugentlige stand-up møder udover vores ugentlige gruppemøde. Stand up-møderne afholdes som udgangspunkt hver mandag og hver fredag direkte efter, at undervisningen er slut. Disse stand-up møder skal  bruges at følge op på, hvad vi hver især i gruppen har arbejdet med siden sidst, vende eventuelle problematikker ift. tvivl eller tidspres, så vi på bedst mulig vis kan hjælpe hinanden inden det er for sent.

Til gruppemøderne er der udarbejdet en dagsorden forinden og der skrives referat af hvert møde. Dagsorden og referat til gruppermøderne er Sarah ansvarlig for. Sarah er endvidere ordstyrer både til gruppemøder og til stand-up møder. 

\subsection{Vejledermøder}
Vejledermøde afholdes som udgangspunkt én gang ugentligt medmindre gruppen ingen spørgsmål har til vejlederen og derfor finder mødet unødvendigt. I sådan et tilfælde aflyses mødet blot og gruppen arbejder videre med projektet i tidsrummet i stedet. Bortset fra de to første vejledermøder er vejledermødet afholdt på et fast tidspunkt - Torsdag kl. 12:00 hver uge. Til møderne går det på skift hvem der er mødeleder og hvem der er referent. Mødelederen har ansvaret for at gennemgå dagsordenen.

Til vejledermøderne er der blevet udformet aktionspunkter, der skulle arbejdes med til næste gang. I nogle tilfælde har aktionspunkterne ikke været helt færdige og i disse tilfælde er punkterne blot fortsat ind i næste sprint. 

Vejledermøderne er afholdt med vejleder, Samuel. Har der været mere konkrete spørgsmål ifb. Hardware/software og ikke projektet som helhed har det været muligt at deltage i ugentlig vejledning hos Thomas Nielsen (Hardware) og Jesper Rosholm Tørresøe (Software). Det har også været muligt at hente hjælp hos Lars Mortensen.

