\section{Samarbejdskontrakt}
Gruppen har i fællesskab ved projektets start valgt at lave en samarbejdskontrakt. Alle i gruppen har på tidligere semestre benyttet sig af en samarbejdskontrakt og vurderer på baggrund af det, at det fungerer godt. Vi har i gruppen valgt at bygge videre på en tidligere anvendt samarbejdskontrakt, der indeholder mange af de vigtige elementer der skal til, for at gennemførelsen af projektet kan foregå optimalt. Dét udbytte vi i gruppen får ved brug af samarbejdsaftalen, er et bedre gruppearbejde. De regler og krav der er stillet til gruppemedlemmerne og til projektet er formuleret og skrevet ned og vil derfor bidrage til, at vi kan undgå diskussioner og forvirring omkring disse, da man altid kan finde samarbejdsaftalen frem. Samarbejdskontrakten er derfor også et godt afsæt i en god måde at håndtere konflikter på såfremt de skulle opstå.

\vspace{0.2 cm}
Samarbejdskontrakten indeholder punkter ift. Møder, herunder gruppemøder og vejledermøder, afbud ifb. møder samt referater fra møder, ledelse og konflikter i projektet, ambitionsniveau, omgangstone, logbog og brud på samarbejdskontrakten.