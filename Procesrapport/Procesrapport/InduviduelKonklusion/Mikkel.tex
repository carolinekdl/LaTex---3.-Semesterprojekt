\section{Mikkel}
Semesterprojektet har været virkeligt godt, da man fik en autentisk følelse af at være en ingeniør i og med man fik lov til at arbejde med et større og mere omfattende projekt. Det var spændende at have et samarbejde med en motiveret gruppe, for til slut at ende ud med et projekt, som vi alle kan være stolte af. Jeg er gennem projektet blevet udfordret, som et medlem af hardwaregruppen, men har fået et stort udbytte af at skulle designe, beregne og bygge vores system. Vi stødte på nogle problemer i løbet af hardwareprocessen, som vi løste ved at lave grundige gennemgange af vores beregninger samt snakke med Thomas og andre grupper. Jeg føler jeg har fået et meget større kendskab særligt til hardware, end hvad jeg havde før vi begyndte projektet.  Dog har det været svært at hjælpe softwaregruppen da vi som hardwaregruppe har haft vores eget fokus og ikke deltaget i alle de softwaremæssige diskussioner.

Derudover var det en god oplevelse, at vi på baggrund af de andre semesterprojekter, vidste præcis hvordan vi procesmæssigt skulle tilgå projektet. Vi har løbende holdt SCRUM-, gruppe- og vejledermøder, hvor der gruppen blev opdateret om hvor langt hvert enkelt medlem var med sit ansvarsområde. Gruppen har været upåklagelig, da vi alle har bidraget til arbejdet om at nå et fælles mål. Der har været nogle gode diskussioner på gruppen, hvor der ud fra konklusioner på disse er taget beslutninger om hvordan vi skulle gribe tingene an. Overordnet set var det et godt og lærerigt projekt.

\clearpage

