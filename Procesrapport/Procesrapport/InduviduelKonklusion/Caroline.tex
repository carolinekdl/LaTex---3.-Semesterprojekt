\section{Caroline}
Dette semester har semesterprojektet været med til at give de enkelte gruppemedlemmer muligheden for at fordybe sig meget mere i enten hardware eller software fordi projektet har været så stort. Vi blev hurtigt enige om at fordelingen skulle være 3 på hardware og 4 på software. Jeg valgte personligt at være en del af hardware-teamet fordi det er det, jeg interreserer mig mest for og dét, som jeg synes er sjovt at arbejde med. Jeg synes overordnet set at projektet har været meget lærerigt og det har været fedt at få lov til at arbejde med noget mere elektronik. Til tider har det været svært fordi vi som ST'ere ikke altid sidder med så meget elektronik i hænderne, og det har derfor været nødvendigt at trække noget glemt viden fra kurser på tidligere semestre. Derfor har der været en del udfordringer undervejs men samtidig er det for mig at se positivt at opfriske disse kurser igen. 

Selve strukturen omkring projektvejledningen har foregået anderledes end de tidligere semestre hvilket jeg synes er rigtig positivt. At alle grupper har haft samme projektvejleder til projektet som helhed og til selve proces-delen af projektet har fungeret rigtig godt fordi der derudover har været mulighed for vejledning i enten hardware eller software på faste ugentlige tidspunkter. Med denne struktur føler jeg at der er en mere rimelig fordeling af vejledninig til grupperne fordi der ikke er nogle, der har en vejleder, som er god til én af delene og nogle, der har en vejleder, som er god til noget andet. På tidligere semestre har der været meget forvirring om at skulle gå til andre vejledere for hjælp eller at gruppens egen vejleder skal forhøre sig andre steder - klart en positiv oplevelse med denne strukturering af projektet, så det synes jeg klart skal fortsætte og eventuelt indføres på de andre semestre også. 

I forhold til gruppen føler jeg vi har haft et rigtig godt samarbejde hvor de, som gerne ville lave meget fik muligheden for at melde sig på til at gøre dette og omvendt. Vi har ikke haft nogle konflikter internt i gruppen undervejs og vores ugentlige planlægning af møder, stand-up-møder og projektledelsen samt arbejdsmetoden scrum har fungeret rigtig godt.

Jeg har personligt lært meget af projektet. Jeg har fået en større viden omkring hardware og jeg synes det har været sjovt at få lov at lære at designe sit eget print, selvom vi måske ikke får brug for dette fremover. Jeg synes kun at arbejdet med hardware har styrket mine interesser inden for området og det er godt når vi senere hen skal til at vælge lidt mere specialiseret hvad det er, vi gerne vil fremover - evt. vedrørende valgfag. 

Jeg valgte sidst men ikke mindst at fordybe mig i LaTex og har derfor været en stor del af rapportopsætningen, processrapporten samt diverse bilag herunder. Det har været sjovt at lære sideløbende med projektet og gør, at vi i gruppen alle føler vi afleverer noget, der ser flot og professionelt ud. 
\clearpage