\section{Mathias}
Dette projekt har lært mig, hvor vanskeligt det er fyldestgørende at dokumentere et udviklingsprojekt. Derudover har jeg også lært, hvor svært det er at være konsekvent med versionsstyring af sine dokumenter. Dette gav sig især til udtryk i slutningen af projektet, da alle dokumenter skulle samles og sættes sammen. Dette minder lidt om min konklusion fra semesterprojekt 2. Derfor vil jeg have særlig meget fokus på dette til næste semesterprojekt. Jeg har også lært, at jeg er god til at opbygge og overskue en større softwareapplikation, hvor der er mange elementer i spil. Jeg er med dette projekt endnu en gang blevet bekræftet i, at software er den del af uddannelsen, som jeg finder mest interessant.
 
Sidste semester konkluderede jeg også, at mine høje forventninger til andre gruppemedlemmer var et problem, da det smittede af på min opfattelse af deres arbejdsindsats. Dette synes jeg er gået bedre denne gang. Jeg har forsøgt ikke at køre solo, som jeg plejer, og give plads til de andre. Jeg har også undgået at påtage mig andres arbejdsopgaver, hvilket også var et problem sidste semester. Jeg synes, dette har haft en god effekt, da jeg ikke tror, nogen i gruppen har set mig som overtrumfende, som måske tidligere har været tilfældet. 

Jeg har således lagt mig mere op ad min Insights Discovery persona som den observerende koordinator. Dette har kunnet lade sig gøre, da vi har haft en scrum master på projektet, som har gjort, at jeg ikke på noget tidspunkt har følt mig ledsaget til at tage en lederrolle. Det har passet mig fint. Dog vil jeg forsøge at blive lidt bedre til at markere i stedet for at tie, når jeg er uenig i beslutninger, der bliver taget i gruppen. 
\clearpage
