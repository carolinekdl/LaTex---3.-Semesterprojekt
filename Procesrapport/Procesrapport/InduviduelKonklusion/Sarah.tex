\section{Sarah Krohn Fenger}
I dette semesterprojekt har jeg haft rollen som projektleder og scrummaster. Dette har været en udfordrende rolle, men også lærerig. Jeg har været glad for, at gruppen har givet rollen til mig. 

Det har været en lærerig proces for mig som person, idet at jeg godt kunne tænke mig, når jeg er færdig uddannet at arbejde som projektleder. Derfor har jeg også igennem dette semesterprojekt kunne få øjenene op for, hvad jeg skal blive bedre til som projektleder. Jeg skal dels være opmærksom på, at det ikke er mit ansvar, at alle opgaver bliver løst. Det er dog min opgave at prioritere, hvilke opgaver der skal laves først. Derudover skal jeg arbejde med, at kritik af et foreslag samt en opgave ikke er en kritik af mig som person. 

På vores 1. semesterprojekt fik jeg også rollen som leder, men til forskel fra det semesterprojekt synes jeg, at jeg er blevet bedre til at skære igennem i beslutningssituationer. Derudover har vi også flere gange haft situationer, hvor ingen har sagt noget, og i disse situtationer er jeg blevet bedre til at holde fast i at nå frem til en beslutning og få de fleste i gruppen til at melde sig på banen. 

Noget jeg dog har fået øjnene op for, som jeg skal arbejde på dels som projektleder, men også som gruppemedlem er, at få snakket med et gruppemedlem, hvis jeg gentagende gange oplever situationer, hvor jeg føler mig provokeret eller irriteret. Jeg har oplevet, at jeg hverken på 1. semesterprojekt eller på dette projekt har reageret, hvilket har gået udover mit humør og engagement i projektet.


Dette semesterprojekt har været et projekt, hvor man virkelig har fået det at være ingeniør ind under neglene på trods af rammerne for projektet var sat. Det har været spændende, at vi både har skulle udvikle en hardware og en softwaredel og få disse to dele til at snakke sammen. Derudover har vejledningen i dette projekt fungeret rigtig godt i og med der har været en vejleder indenfor hver del; hardware, software og proces. 

\clearpage