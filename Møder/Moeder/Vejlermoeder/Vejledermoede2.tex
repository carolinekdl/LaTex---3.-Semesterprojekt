\section{Vejledermøde 2}

\vspace{0.5 cm}
\textbf{Indkaldelse til vejledermøde \#2} \\

Dato: 18-09-2018 \\
Tid: 12:00 \\
Sted: 408E \\
Deltagere: () Til stede, (F) for sent, (A) meldt afbud, (U) Udeblev \\
Caroline(), Kajene(), Mathias(), Mikkel(), Nicolai(), Sarah(), Thea(F), Samuel() 

\vspace{0.1 cm}
\textbf{Dagsorden:}

\begin{enumerate}
	\item Valg af mødeleder
	\item Valg af referent
	\item Godkendelse af referat fra forrige møde
	\item Opfølgning af aktionspunkter fra forrige møde
	\item Spørgsmål til Samuel
	\item Gennemgang af tidsplan
	\item Nye aktionspunkter til næste møde
	\item Tidspunkt for næste møde
	\item Evt.
\end{enumerate}

\textbf{Referat:}

\begin{enumerate}
	\item Caroline
	\item Kajene
	\item Godkendt
	\item Valg af projektleder: Sarah \\
	Tidsplan færdiggøres (Sarah) \\
	Scrum: Aftalt stand-up møder 2x ugentligt ud over vejledermøde og gruppearbejde. Stand-up møderne afholdes som udgangspunkt mandag og fredag. Sprint fra onsdag-onsdag (Ugentligt gruppearbejde hver onsdag 8.15-9.50)\\
	Færdiggøre og rette kravspecifikation og accepttest. Sendes til Samuel. Navne på alle opgaver (opdater Pivotal Tracker)\\
	Timeplanen er færdiggjort på TeamGantt, men den skal ikke følges slavisk. Der kan senere komme rettelser og justeringer. \\
	
	\item \textbf{Standard 60601-1-8 tabel 3, 4, 5 samt punkt 6.1} \\ tabel 3: high, medium og low. Standarder alt efter hvor vigtigt situationen er. Vi skal have lavet nogle toner i matlab, og fundet ud af om tonen skal være medium eller high. Man skal justere alt dette efter moscow. \\
	tabel 5: mute og off, hvor lang tid den er pauset i eller slukket \\
	symboler på knapperne er en god usability, visuelt nemmere logfil er en god ting at have, da man kan gå ind og se på hvilke filer den holder på osv. blokering osv \\
	man kan risikerer at have 5 alarmer samtidig, og der skal man bruge softwaretest, som gør at man kan håndtere alarmerne \\
	
	\textbf{Nulpunktsjustering} \\ Nulpunktsjustering SKAL stå rigtigt (Vis evt et billede eller video). Den blå ting, sørger for at der er flow igennem, sådan at der ikke er bobler inde røret. Luften forbundet til sensor, væske forbundet til sensor eller er der lukket helt af. Sætter man nulpunktsjusttering igang når der er væske, så vil der være et højt tryk derinde. \\
	
	\textbf{Ikke-funktionelle krav:} \\ der skal være tal på, INGEN bullets. Krav 1 og krav 2 skal ikke med accepttesten, hellere ved integrationstesten \\
	
	\item Godkendt \\
	\item - \\
	\item Næste møde: Tirsdag d. 25-09-2018 kl. 12:00 \\ 
	\item \textbf{Kommentarer til kravspecifikation og accepttest:} \\ ved radiobutton skal der være valgt default. Defaulten skal vi selv beslutte
	nulpunktsjustering skal være ved brugergrænsefladen \\ bluetooth skal beskrives nede ved aktørbeskrivelsen også \\
	kalibrering skal også forbindes til blodtrykssensoren ved usecasediagrammet \\
	Hovedscenariet Fullydressed use case: Extension skal beskrives mere nøjagtigt. Der skal tages stilling til om der signal ind. Vi får ikke noget brugbart signal, eller for lille til at kunne registreres \\
	Usecase 2: Nulpunktsjustering skal ikke være i denne use case. Man kan ikke lave en blodtryksmåling før der er foretaget en nulpunktsjustering. Man skal først trykke på “start”-knappen \\
	Juster grænseværdier usecase: hvilke grænseværdier? talværdier? \\
	Alarmering UC: Det er uklart hvad der menes. Extensions skal rettes. \\
	Slet de steder hvor der står Visual Studio, skriv hellere system \\
	Afstand skal ændres til 1.5 meter ved ikke-funktionelle krav \\
	ikke-funktionelle krav, udover should have, skal ikke være i accepttesten, da vi ikke tester dem \\
	Hvad menes der med alarmer? Hvor og hvordan skal den alarmere?
	
\end{enumerate}

\clearpage