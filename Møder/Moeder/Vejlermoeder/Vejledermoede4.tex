\section{Vejledermøde 4}

\vspace{0.5 cm}
\textbf{Indkaldelse til vejledermøde \#4} \\

Dato: 12-10-2018 \\
Tid: 13:00 \\
Sted: 408E \\
Deltagere: () Til stede, (F) for sent, (A) meldt afbud, (U) Udeblev \\
Caroline(), Kajene(A), Mathias(A), Mikkel(), Nicolai(), Sarah(), Thea(F), Samuel() 

\vspace{0.1 cm}
\textbf{Dagsorden:}

\begin{enumerate}
	\item Valg af mødeleder
	\item Valg af referent
	\item Godkendelse af referat fra forrige møde
	\item Opfølgning af aktionspunkter fra forrige møde
	\item Spørgsmål til Samuel
	\item Gennemgang af tidsplan
	\item Nye aktionspunkter til næste møde
	\item Tidspunkt for næste møde
	\item Evt.
\end{enumerate}

\textbf{Referat:}

\begin{enumerate}
	\item Mikkel
	\item Sarah
	\item Godkendt
	\item Aktionspunkter opfulgt
	\item \textbf{Blodtryk og puls:} \\
	Systolisk er max og diastolisk er min. Man kan finde top og bund af signalet. 
	Hvor langt tid skal man kigge tilbage? Minimum én cyklus tilbage. Når blodtrykket falder vil der komme et delay. Dette skal vi beslutte. Overvej om vi evt. bruge pulsen til at kigge over fx tre pulsslag - men så skal vi kunne finde den. Hvis den fejler, så kan det være vi skal have nogle min eller maximum. 
	
	Når vi skal finde puls finder vi toppunkter. Hvis vi finder toppunkter og bundpunkter kan vi tage blodtryksværdier ud til disse punkter. Der kan dog komme et krav om støj. \\
	
	\textbf{Sundhedsfagligt:} \\
	Vi skal skrive det sundhedsfaglige som vi mener skal være med dvs. passer til vores brugssituation. Fx hjerteklapper.
	
	\textbf{HW:} \\
	Rettelser til omkring review gruppe ift. vejledning. Har snakkede omkring strømforsyning. \\
	
	\textbf{SW:} \\
	Review med anden software. Snakke efter review omkring det. Vi vil lave diagrammer færdig. \\
	
	\item Godkendt \\
	\item - \\
	\item Næste møde: 25-10-2018 kl. 12:00 \\ 
	\item -
	
\end{enumerate}

\clearpage